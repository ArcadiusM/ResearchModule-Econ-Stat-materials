%template Master Thesis 
%University of Bonn Master of Life Science Informatics
% arara: pdflatex: { synctex: on }

\documentclass[ twoside=false, 12pt,  footinclude=true,  headinclude=true,  cleardoublepage=empty]{scrbook}

\usepackage[utf8]{inputenc}
\usepackage [english] {babel} 

\usepackage[]{biblatex}
\addbibresource{references.bib}

\usepackage{lipsum}
\usepackage[linedheaders,parts,pdfspacing]{classicthesis}
\usepackage{amsmath}
\usepackage{amsthm}
\usepackage{booktabs}
\usepackage{graphicx}
\usepackage{float}
\usepackage{indentfirst}
\usepackage [T1]{fontenc}
\usepackage{listings}
\usepackage{color}
\usepackage{multirow}
\usepackage{tikz}
\usepackage[toc,page]{appendix}
\usepackage{MnSymbol}
\usepackage{longtable}
\usepackage{graphicx}
\usepackage{subcaption}
\usepackage{mathtools} 
\usepackage{enumerate}
\usepackage{csquotes}
\usepackage{amsmath}
\usepackage{hyperref}
\usepackage{acro}
\usepackage[a4paper,includeall,margin=2cm,marginparsep=0cm,marginparwidth=0cm,left=3cm,right=2cm,top=2cm,bottom=2cm]{geometry}
\usepackage[font={footnotesize,it}, labelfont=bf]{caption}

\setlength{\parskip}{10pt} % Abstand zwischen den Absätzen
\setlength{\parindent}{0pt} %Erstzeileneinzug 
\setcounter{secnumdepth}{3}
\setcounter{tocdepth}{4}

\DeclareAcronym{AD}{
	short = AD ,
	long  = Alzheimer's disease
}

\title{Master Thesis}
\date{\today}
\begin{document}
	\begin{titlepage}
		\centering
		University of Bonn
		
		 Master Programme in Economics
		\vspace{1in}
		\vspace{1in}
		
		{\LARGE \bfseries  Random Forest for Classification Problems}
		\vspace{1in}
		
		{\large Submitted by}
		
		{\LARGE Burak Balaban \par
				Arkadiusz Modzelewski\par
				Raphael Redmer\par}
		
		\vspace{1in}
		
			Supervisor: Prof. Dr. Dominik Liebl
			
		\vfill
		
		\begin{flushleft}
			\today
		\end{flushleft}
		
	\end{titlepage}
	
%	% Add blank page
%	\newpage
%	\thispagestyle{empty}
%	\mbox{}
	
	% /frontmatter -> Turn on roman numbering for the following content and turns off normal numbering
	
	\frontmatter

\tableofcontents

\chapter*{Abstract}
As a non-parametric estimation tool, decision trees attract attention in the economics literature. Yet, decision trees suffer from high variance and, for prediction purposes higher variance seems to be a crucial problem, thus, several improvements were proposed such as bootstrap aggregation, boosting and most importantly random forests. In this project, while the main focus is being on the random forest, The elements of statistical learning by \cite{friedman2001elements} \cite{varian2014big} \cite{maimon2005data}, \cite{louppe2014understanding} and as expected \cite{breiman2001random} are the main literature that will be utilized in this project.

To explain the concept of random forests in full extent, primarily decision trees should be discussed. Exploiting the main idea and struggles with bias-variance trade-off, random forests' importance can be emphasized as a more stable prediction tool \cite{maimon2005data}. Conceptual comparison of random forests with bagging and boosting can deliver a better understanding of its unique features as \cite{lee2019bootstrap} shows in a similar fashion. To get a further understanding, random forests’ estimation process can be mathematical explained \cite{biau2012analysis} and likewise, examining the consistency of estimator and showing the properties can be included \cite{breiman2004consistency}, \cite{denil2014narrowing}. Also, variable importance in the tree growing process is another area that needs to be delved into \cite{ishwaran2007variable} and \cite{louppe2013understanding}.



% /frontmatter -> Turn on normal numbering 
\mainmatter

\chapter{Introduction}
\label{ch:intro}

Lorem ipsum dolor sit amet, consetetur sadipscing elitr, sed diam nonumy eirmod tempor invidunt ut 
labore et dolore magna aliquyam erat, sed diam voluptua. At vero eos et accusam et justo duo 
dolores et ea rebum. Stet clita kasd gubergren, no sea takimata sanctus est Lorem ipsum dolor sit amet.
Lorem ipsum dolor sit amet, consetetur sadipscing elitr, sed diam nonumy eirmod tempor invidunt ut labore
et dolore magna aliquyam erat, sed diam voluptua. At vero eos et accusam et justo duo dolores et ea rebum.
Stet clita kasd gubergren, no sea takimata sanctus est Lorem ipsum dolor sit amet.

\chapter{Decision tree}
Lorem ipsum dolor sit amet, consetetur sadipscing elitr, sed diam nonumy eirmod tempor invidunt ut labore
et dolore magna aliquyam erat, sed diam voluptua. At vero eos et accusam et justo duo dolores et ea rebum.
Stet clita kasd gubergren, no sea takimata sanctus est Lorem ipsum dolor sit amet.

\section{Main idea and illustration}
Lorem ipsum dolor sit amet, consetetur sadipscing elitr, 
sed diam nonumy eirmod tempor invidunt ut labore et dolore magna aliquyam erat, sed diam voluptua.
At vero eos et accusam et justo duo dolores et ea rebum. Stet clita kasd gubergren,
no sea takimata sanctus est Lorem ipsum dolor sit amet.

\section{Mathematical explanation}
Lorem ipsum dolor sit amet, consetetur sadipscing elitr,
sed diam nonumy eirmod tempor invidunt ut labore et dolore magna aliquyam erat, sed diam voluptua.
At vero eos et accusam et justo duo dolores et ea rebum. Stet clita kasd gubergren,
no sea takimata sanctus est Lorem ipsum dolor sit amet.

\section{Splitting criteria}
Lorem ipsum dolor sit amet, consetetur sadipscing elitr, 
sed diam nonumy eirmod tempor invidunt ut labore et dolore magna aliquyam erat, sed diam voluptua.
At vero eos et accusam et justo duo dolores et ea rebum. Stet clita kasd gubergren,
no sea takimata sanctus est Lorem ipsum dolor sit amet.

\section{Bias-variance trade-off}
Lorem ipsum dolor sit amet, consetetur sadipscing elitr, 
sed diam nonumy eirmod tempor invidunt ut labore et dolore magna aliquyam erat, sed diam voluptua.
At vero eos et accusam et justo duo dolores et ea rebum. Stet clita kasd gubergren,
no sea takimata sanctus est Lorem ipsum dolor sit amet.

\subsection{Bagging and boosting}
Lorem ipsum dolor sit amet, consetetur sadipscing elitr, 
sed diam nonumy eirmod tempor invidunt ut labore et dolore magna aliquyam erat, sed diam voluptua.
At vero eos et accusam et justo duo dolores et ea rebum. Stet clita kasd gubergren,
no sea takimata sanctus est Lorem ipsum dolor sit amet.


\chapter{Random Forest}
Decision trees, which we mentioned in the previous section, have been used for a long time. Deployment of decision trees is visible in simple situations and also in more complex scientific or real life and industrial affairs. The recent popularity of decision trees is due to work presented by Breiman between 1996 and 2004 that ensamples of different decision trees can get a meaningful improvement in accuracy in classification problems and other common learning tasks such as regression. As the unit in this procedure is based on decision tree and includes an injection of randomness, this method is known as a random forest. 

\section{Main idea and illustration}
An ensemble of randomly trained decision trees, so in other words random decision forest was defined by Breiman in his work ,,Random Forests”  from 2001 as follows:

\newtheorem{theorem}{Definition}
\begin{theorem}
A random forest is a classifier consisting of a collection of tree-structured classifiers \{${h(\textbf{x},\Theta_{k})}, k = 1,2,...$\} where the \{$\Theta_{k}$\} are independent identically
distributed random vectors and each tree casts a unit vote for the most popular class at input $\textbf{x}$ .
\end{theorem}
The main aspect of a random decision forest model is an injection of randomness which allows to have all unit trees different from the others. Two key concepts that makes decision forest "random" are:
\begin{enumerate}
\item Random sampling of training data points when building trees
\item Random subsets of features considered when splitting nodes
\end{enumerate}
Randomness parameter has a meaningful impact on the model, because except controlling the amount of randomness within each tree, it controls also the amount of correlation between different trees in the forest. As randomness parameter decreases, trees become more decorrelated [Decision Forests: A Unified Framework for Classification, Regression, Density Estimation, Manifold Learning and Semi-Supervised Learning]. Training of all trees is done independently and testing consists in the fact that each test point is pushed through every tree included in forest until it ends in corresponding leaves. As a last step is taking all predictions from every unit decision tree and combining them into prediction of single random forest. Combination of different tree predictions can be done in different ways. In random forests for classification problems, forests generate probabilistic output. It means that they return not just a single class point prediction, but whole class distribution, so combination of tree predictions can be described as below:

\begin{center}
$p(c|\textbf{v}) =  \frac{1}{T} \displaystyle\sum_{t}^{T} p_{t}(c|\textbf{v})$
\end{center}

where in a random forest with the number of decision trees equals to $T$ each tth tree obtains the posterior distribution $ p_{t}(c|\textbf{v})$. The class label here is symbolized by $c$ such that $c \in \textbf{C}$ with $ \textbf{C} = \{ c_{k}\} $



\section{Mathematical explanation and Consistency}


\section{Interpretation}


\section{Variable importance}

\section{Error estimation for random forest}



\chapter{Application and Comparison}
Lorem ipsum dolor sit amet, consetetur sadipscing elitr, sed diam nonumy eirmod tempor invidunt ut labore
et dolore magna aliquyam erat, sed diam voluptua. At vero eos et accusam et justo duo dolores et ea rebum.
Stet clita kasd gubergren, no sea takimata sanctus est Lorem ipsum dolor sit amet.

\section{Application of random forest}
Lorem ipsum dolor sit amet, consetetur sadipscing elitr, sed diam nonumy eirmod tempor invidunt ut labore
et dolore magna aliquyam erat, sed diam voluptua. At vero eos et accusam et justo duo dolores et ea rebum.
Stet clita kasd gubergren, no sea takimata sanctus est Lorem ipsum dolor sit amet.

\subsection{Simulated data}
Lorem ipsum dolor sit amet, consetetur sadipscing elitr, sed diam nonumy eirmod tempor invidunt ut labore
et dolore magna aliquyam erat, sed diam voluptua. At vero eos et accusam et justo duo dolores et ea rebum.
Stet clita kasd gubergren, no sea takimata sanctus est Lorem ipsum dolor sit amet.

\subsection{Real data example}Lorem ipsum dolor sit amet, consetetur sadipscing elitr, sed diam nonumy eirmod tempor invidunt ut labore
et dolore magna aliquyam erat, sed diam voluptua. At vero eos et accusam et justo duo dolores et ea rebum.
Stet clita kasd gubergren, no sea takimata sanctus est Lorem ipsum dolor sit amet.

\section{Gradient Boosting Classifier}Lorem ipsum dolor sit amet, consetetur sadipscing elitr, sed diam nonumy eirmod tempor invidunt ut labore
et dolore magna aliquyam erat, sed diam voluptua. At vero eos et accusam et justo duo dolores et ea rebum.
Stet clita kasd gubergren, no sea takimata sanctus est Lorem ipsum dolor sit amet.

\section{AdaBoost Classifier}
Lorem ipsum dolor sit amet, consetetur sadipscing elitr, sed diam nonumy eirmod tempor invidunt ut labore
et dolore magna aliquyam erat, sed diam voluptua. At vero eos et accusam et justo duo dolores et ea rebum.
Stet clita kasd gubergren, no sea takimata sanctus est Lorem ipsum dolor sit amet.

\chapter{Conclusion and outlook}

Lorem ipsum dolor sit amet, consetetur sadipscing elitr, 
sed diam nonumy eirmod tempor invidunt ut labore et dolore magna aliquyam erat, sed diam voluptua.
At vero eos et accusam et justo duo dolores et ea rebum. Stet clita kasd gubergren,
no sea takimata sanctus est Lorem ipsum dolor sit amet.


\printbibliography


\chapter*{Declaration}
I hereby certify that this material is my own work, that I used only those sources and resources referred to in the thesis, and that I have identified citations as such.
		
\vspace{0.3in}

\noindent Bonn, \today

\end{document}
